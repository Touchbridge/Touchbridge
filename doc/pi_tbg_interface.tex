\documentclass[12pt]{article}
\usepackage{booktabs} % Better horizontal rules in tables
%\usepackage{ebgaramond}
\usepackage{fbb}
\usepackage{float}
\usepackage{tikz-timing}
\usepackage{bytefield}
\pagestyle{empty}

% Make a command to show an overbar for logic inversion
\newcommand{\inv}[1]{$\overline{\mbox{#1}}$}
\newcommand{\outdir}[0]{$\rightarrow$}
\newcommand{\indir}[0]{$\leftarrow$}
\newcommand{\bidir}[0]{$\leftrightarrow$}

\begin{document}

\title{Touchbridge Specification}
\author{James Macfarlane\\james@ael.co.uk}
\maketitle

\section{Touchbridge Interface Specification for Raspberry Pi HAT}

\begin{table}[!htb]
    \centering

    \begin{tabular}[pos]{l l l l l}
    \toprule
        Function        & Pi 2/B+ Pin  & Direction & STM32F103 Pin (LQFP48) & (LQFP64) \\
    \midrule
        D0              & GPIO16  & \bidir      & PA0  & PC0  \\
        D1              & GPIO17  & \bidir      & PA1  & PC1  \\
        D2              & GPIO18  & \bidir      & PA2  & PC2  \\
        D3              & GPIO19  & \bidir      & PA3  & PC3  \\
        D4              & GPIO20  & \bidir      & PA4  & PC4  \\
        D5              & GPIO21  & \bidir      & PA5  & PC5  \\
        D6              & GPIO22  & \bidir      & PA6  & PC6  \\
        D7              & GPIO23  & \bidir      & PA7  & PC7  \\
        ADSEL           & GPIO24  & \outdir     & PA8  & PC8  \\
        WRSEL           & GPIO25  & \outdir     & PA9  & PC9  \\
        \inv{EN}        & GPIO26  & \outdir     & PA10 & PC10 \\
        \inv{ACK}       & GPIO27  & \indir      & PA11 & PC11 \\
        \inv{INT}       & GPIO12  & \indir      & PA12 & PC12 \\
        Spare           & GPIO13  &             &      & PC13 \\
    \bottomrule
    \end{tabular}

    \caption{Pin assignments for Raspberry Pi Touchbridge HAT}\label{tab:a}
\end{table}

\begin{figure}[H]
    \centering

    \begin{tikztimingtable}[ %
        timing/dslope=0.1,
        timing/.style={x=6ex,y=2ex},
        x=5ex,
        timing/rowdist=3ex,
        timing/name/.style={font=\rmfamily\scriptsize}
    ]
        D[7..0]         & D {} 6D{Data From Pi} {} D\\
        ADSEL           & D {} 6D{Data From Pi} {} D\\
        WRSEL           & D {} 6H {} D\\
        \inv{EN}        & 2H 3L  3H\\
        \inv{ACK}       & 4H 2L  2H\\
        \\
        STM32 Read Event & 3L G  5L\\
    \extracode
      \fulltablegrid
    \end{tikztimingtable}

    \caption{Write Timings}\label{tab:a}
\end{figure}

\begin{figure}[H]
    \centering

    \begin{tikztimingtable}[ %
        timing/dslope=0.1,
        timing/.style={x=6ex,y=2ex},
        x=5ex,
        timing/rowdist=3ex,
        timing/name/.style={font=\rmfamily\scriptsize}
     ]
        D[7..0]         & D {} 2Z {} 4D{Data from STM32} {} 2Z {} 1D\\
        ADSEL           & 1D {} 8D{Data from Pi} {} 1D\\
        WRSEL           & 1D {} 8L {} 1D\\
        \inv{EN}        & 2H 4L  4H\\
        \inv{ACK}       & 4H 4L  2H\\
        \\
        Pi Read Event   & 5L G  5L\\
    \extracode
      \fulltablegrid
    \end{tikztimingtable}

    \caption{Read Timings}\label{tab:a}
\end{figure}

\iffalse
\begin{figure}[H]
    \centering
    \begin{bytefield}[endianness=big,bitwidth=2.1em]{9}
        \bitbox[]{1}{FS} & \bitbox[]{8}{Data Bus} \\
        \bitheader{0,3,4,7} \\
        \bitbox{1}{1} & \bitbox{4}{Type} & \bitbox{4}{Length} \\
        \bitbox{1}{0} & \bitbox{8}{Payload Byte~0} \\
        \bitbox{1}{0} & \bitbox{8}{Payload Byte~1} \\
        \wordbox[]{1}{$\vdots$} \\[1ex]
        \bitbox{1}{0} & \bitbox{8}{Payload Byte~$N$}
    \end{bytefield}
    \caption{Frame Protocol}
\end{figure}
\fi

\section{Touchbridge Protocol}

\subsection{Introduction}

Touchbridge is physically implemented using a CAN (Controller Area Network)
bus. CAN bus was originally designed to for use in automobiles to allow various
electronic control units to talk to each other but it has many other
applications. The CAN protocol (and hardware which implements it) provides a
nice method of dealing with message collisions (when two devices try to send at
the same time.) This means that it is a fairly reliable way of getting messages
from one device to another on a bus. However, the CAN protocol doesn't dictate
how the messages are to be used. This means that another protocol layer is
usually placed on top of CAN in order to make it useful for a particular
application. Touchbridge adds a light-weight protocol layer on top of CAN bus
which gives it a lot of useful extra capabilities.

During the design of the Touchbridge system, existing CAN wrapper layers where
considered but all where found to be some combination of too
application-specific, too proprietary or too combersome to implement (at least
without paying \$\$\$\$ for a commerical SDK.) Hence the need for something
new: the Toucbridge Protocol. Although the Touchbridge Controller API
(described eslewhere) hides the low-level stuff from the end-user, one of the
aims of the Touchbridge Protocol is that it be simple enough that most of
functionality of the Touchbridge peripherals is available to a controller which
can send and receive simple CAN bus messages one at a time, without needing to
write any kind of protocol stack. In other words, most of the hard stuff is
done on the peripherals, not the controller.


\subsubsection{CAN Bus Messages}

This document is not supposed to be an exhaustive primer on CAN bus - there are
better sources available elsewhere. The following information is presented as a
very brief introduction or aide memoire but it should be sufficient to
understand the internals of Touchbridge. 

A CAN message contains three main parts: an ID field, a length field and a data
field. The data field can be up to 8 bytes long. The length field says how many
bytes are present in the payload. The ID field is commonly used to say how the
data field should be interpreted or to do addressing of one form or another.
The CAN protocol doesn't place any constraints on how the ID field should be
used. This is left up to the application. CAN supports two different lengths of
ID field, Standard (11 bits long) and Extended (29 bits long.) Touchbridge uses
the Extended form. The CAN message also has an RTR bit (not used in
Touchbridge, always set to zero) and an IDE bit which is set if the extended ID
format is in use (always set to one in Touchbridge.)

\subsubsection{Nodes}


In Touchbridge, a node means a physical device attached to the bus and has its
own unique address. The Touchbridge node address is represented by a 6-bit
number. This means that 64 addresses are possible. However, address 0 is
reserved and address 63 is the broadcast address (used for sending a messsage
to all nodes at once) which leaves 62 addresses free.  This is the same as the
maximum recommended number of nodes on a CAN bus segment due to electrical
limitations. Because many more than 62 physical touchbridge devices can exist
in the world, a method is needed of assigning the limited number of node
address to the devices present on a particular bus. Touchbridge has two methods
of doing this: Dynamic Assignment and Static Assginment. They are
covered in the section called "Address Assignment."

\subsubsection{Controllers and Peripherals}

In the way Touchbridge is commonly used, nodes can be thought of either as
controllers (something running user's code, e.g. a Raspberry Pi with a
Touchbridge "HAT") or peripherals (something which does useful work, e.g.
controls a motor or reads data from a sensor.) There is usually only one
controller node per Touchbridge bus, though it is possible to have multiple
controllers (e.g. for redundancy.)

Because Touchbridge peripherals all contain microcontrollers it is possible to
blur the lines between what is a controller and what is a peripheral by loading
custom firmware. While this defeats much of the flexibility that Touchbridge
was designed for, it can nonetheless be useful in some applications which are
very sensitive to cost, power or weight, or require very high reliablity.

\subsubsection{Request and Response Messages}

Touchbridge nodes send and receive CAN bus messages in accordance with
the Touchbridge Protocol as specified below. Touchbridge defines 4 different
message types: Reqeust, Response, Indication and Error Response.

The most common message types are Request and Response. These allow a
transaction to take place between nodes as follows. Node A, which we'll call
the requesting node, sends a Request message to node B, which we'll term the
responding node. Node B sends a Response message back to node A. Normally the
requesting node is a controller and the responding node a peripheral, though
there can be exceptions to this rule.

Touchbridge uses the request-response transaction to execute remote procedure
calls (RPCs) which perform useful actions. The request message contains the
identifier of the function to be called, along with any parameters it needs.
The responding node receives the request message and executes the required
function. It then sends a response back to the requesting node which contains
the results (if any) of executing the function. Each node can support multiple
functions and different types of node will make different functions available.

For example, a controller node might send a message to a peripheral node which
has high-current output drivers for controlling lamps or motors (e.g. the
TBG-HSO card.) The request message calls a function which controls digital
outputs.  The message data field contains a bit mask and a value which says
which outputs are to be affected and whether they should be on or off. The
peripheral node aplies the changes to its output pins and sends a response
message. In this case, the response has no data fields, it simply acknowledges
that the function has returned propely.

Another example, this time for a peripheral with analogue inputs (e.g. the
current sensors on the TBG-HCO card.) The request message contains the function
to be called (in this case, the "analogue read" functiona) and the channel
number of the desired input. The function triggers the peripheral's analogue to
digital converter, waits for the conversion to complete and then sends a
response message containing the 16-bit value of the select analogue channel.

Request messages are normally sent to a single node. However, broadcast
requests are possible. Multiple nodes will then try to reply at once.
Forunately, CAN provides a method of coordinating this potentially chaotic
scenario by ensuring that the most "dominant" message gets through first. Put
simply, this is the one with the most zeros near the front of the message.
Broadcast Requests are used to implement dynamic address assignment.


\subsubsection{Indication and Error Response Messages}

In addition to Reqeuest and Response messages, Touchbridge has two further
message types - Indication and Error Response.

An Error Response message is sent instead of a normal response if the
responding node does not know how to interpret a request (e.g. because it
doesn't support the requested function or the parameters passed to a function
where incorrect.)

Indication messages are like request messages which do not expect a response.

When a transaction is not required (i.e. when the requester does not need
acknowledgment that the request has been received) indication messages may be
used in place of request messages to be used to reduce traffic on the bus.

Indication messages sent to the broadcast address can be used notifying all
nodes of an event, such as a timer tick for triggering data sampling.

They can also be used for "unreliable" asyncronous event notification from
periphel back to the controller. For "reliable" event notification,
transactions can be used where the peripheral is the requester and the
controller the responder.



\subsection{CAN Message ID}

\begin{figure}[H]
    \centering
    \begin{bytefield}[endianness=big,bitwidth=1.1em,bitheight=\widthof{~Resvd~},
                boxformatting={\centering\small}
       ]{29}
        \bitheader[endianness=big]{28,27,26,25,24,23,12,11,0} \\
        \bitbox{2}{Type} & \bitbox{1}{\rotatebox{90}{Resvd}} & \bitbox{1}{\rotatebox{90}{Cont}} & \bitbox{1}{\rotatebox{90}{State}} & \bitbox{12}{Dst Addr} & \bitbox{12}{Src Addr} \\
    \end{bytefield}
    \begin{bytefield}[endianness=big,bitwidth=1.1em,
                boxformatting={\centering\small}
       ]{12}
        \bitheader[endianness=big]{11,6,5,0} \\
        \bitbox{6}{Node Addr} & \bitbox{6}{Port} \\
    \end{bytefield}
    \caption{TBG CAN ID \& Address Bit Fields}
\end{figure}


\subsubsection{Type field}

    The type field is 2 bits wide and can have the following values.

\begin{table}[H]
    \centering

    \begin{tabular}[pos]{l l l l}
    \toprule
        Type    & Transaction           & Direction \\
    \midrule
        0       & REQUEST               & \outdir  \\
        1       & RESPONSE              & \indir   \\
        2       & ERROR RESPONSE        & \indir   \\
        3       & INDICATION            & \bidir   \\
    \bottomrule
    \end{tabular}

    \caption{Touchbridge Message Types}\label{tab:tbg_msg_types}
\end{table}

\subsubsection{Continuation (Cont) bit}

    The continuation bit is set if the message contains data which is to follow
on from data in a previous message. This is useful when a remote function needs
more than 8 bytes of data.

    If the continuation bit is not set, the message is either self-contained or
it is the first message of a multi-part set (i.e. the following messages will
have their continuation bit set.) Touchbridge does not explicitly have a way of
indicating the last message in a multi-part set. This means there must be
another way of knowing when the last message has been received - either by the
receiver waiting for a set number of bytes (the number of which is known by the
sender) or by explicitly stating the total message length by using a field in
the first message.

\subsubsection{State bit}

    The state bit allows reliable transactions to be carried out. A transaction
consists of a request message sent from node A to node B, followed by an
response message sent from node B back to node A. In some cases, it is
desirable that an action be definitely carried out by B. In this case, A can
keep sending requests to B until a response is returned. Furthermore, it is
sometimes desireable that an action definitely be carried out by B but only
once. An example of when such behaviour is required is when A calls a function
on B that needs a multi-part message.  Because the function will only work
properly when all the parts of the message have been received, a reliable
transaction must be used for each part. Reliable transactions are also useful
when the node is controlling certain types of device, e.g. one which dispenses
some ammount of a physical material (e.g. products from a vending machine.).
Without reliable transactions, it is possible that a minor network glitch could
cause a response packet to go missing. This could cause the caller to make
multiple requests resulting in too much stuff being dispensed.

The state bit allows reliable transaction to be achieved and it works like
this: A sends a request with the state bit set to 0. Some of these request
could potentially go missing. When B finally receives the request, it carries
out the requested action and sends a response with the state bit set the same
as in the request.  It then inverts the value of the state bit and saves it.
Any further messages which do not match the new state are acknowledged but do
not cause a new action to occur. Some of the responses may potentially go
missing. A keeps sending requests with the same value of the state bit until it
receives a response with the same state as in the original request. It then
toggles the state value used in the next request.

Because state must be saved on the nodes which are taking part in a reliable
transaction and furthermore this state is specific to the pair of nodes which
are partaking, resources must be allocated on the nodes. There must also be a
way to tell the nodes whether or not reliable transactions should be used for a
particular message. Therefor a set-up stage is required in order to enable
reliable transactions and to specify which nodes and ports are taking part in
it. In order to avoid excessive resources being used on peripheral nodes,
reliable transactions are set-up as follows:

For incoming messages: each active port on a device has its own local state bit
and an enable bit.  A command can be sent to the device to enable reliable
transactions on a particular port.

For outgoing messages: each active port that supports outgoing asynchronous
messages has a destination address register. If set, reliable transactions will
be used instead of asynchronous indication messages.

Note that for the re-transmission aspect to work, a re-try counter is required
per outgoing port and a retry limit needs to be defined.

It is not planned to implement reliable transactions in the first version of
the Touchbridge protocol. Implementation would probably require a 'reliable
message' queue on the nodes. It would also make sense to make that queue of
buffers which support multi-part messages as these are one of the main
use-cases of reliable transactions.


\begin{figure}[H]
    \centering
    \begin{bytefield}[endianness=big,bitwidth=\widthof{Overflow},bitheight=\heightof{\vbox{RX\\Data\\Avail\\IE\\~}},
                boxformatting={\centering\small}
       ]{8}
        \bitheader[endianness=big]{7,4,3,2,1,0} \\
        \begin{rightwordgroup}{Status (RD)}
            \bitbox{4}{Address} & \bitbox{1}{Reserved} & \bitbox{1}{\vbox{RX\\Buf\\Overflow}}  & \bitbox{1}{\vbox{RX\\Data\\Available}} & \bitbox{1}{\vbox{TX\\Buf\\Empty}}
        \end{rightwordgroup} \\
        \begin{rightwordgroup}{Config (WR)} \\
            \bitbox{4}{Address} & \bitbox{1}{Reserved} & \bitbox{1}{\vbox{RX\\Buf\\Overflow\\Reset}}  & \bitbox{1}{\vbox{RX\\Data\\Available\\IE}} & \bitbox{1}{\vbox{TX\\Buf\\Empty\\IE}}
        \end{rightwordgroup} \\
    \end{bytefield}
    \caption{TBG RPI Config/Status/Address Register (ADSEL=1) Fields}
\end{figure}

\begin{table}[H]
    \centering

    \begin{tabular}[pos]{l l l l}
    \toprule
        Value   & Address \\
    \midrule
        0       & CAN Message Buffer    \\
        1       & CAN Address Filter 1  \\
        2       & CAN Address Filter 2  \\
        3       & Config Register 1     \\
    \bottomrule
    \end{tabular}

    \caption{TBG RPI Address Map}\label{tab:tbg_rpi_addr}
\end{table}

\end{document}

